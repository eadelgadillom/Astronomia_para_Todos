\documentclass[12pt]{article}
\usepackage[spanish]{babel}
\usepackage{geometry}
\usepackage{amsmath}
\usepackage{authblk} % Paquete para manejar autores y afiliaciones en 'article'
\usepackage{hyperref}
\usepackage{fancyhdr} % Paquete para encabezados y pies de página
\usepackage{colortbl} % Paquete para colorear celdas en tablas
\usepackage{multirow} % Paquete para combinar filas en tablas
\usepackage{booktabs} % Paquete para mejorar el aspecto de las tablas
\usepackage{graphicx}

% Quitar la sangría en todo el documento
\setlength{\parindent}{0pt}

\geometry{letterpaper, margin=1in}

\title{\textbf{Astronomía para Todos}\\ \small Grupo 2\\ 2024-2}

\author[1]{\raggedright Eduardo A. Delgadillo Monsalve}

\affil[1]{\raggedright\small MSc estudiante - Astronomía (eadelgadillom@unal.edu.co) \newline \small Observatorio Astronómico Nacional, Universidad Nacional de Colombia}
\date{}

\pagestyle{fancy}
\fancyhf{} % Limpia los encabezados y pies
\fancyhead[L]{Astronomía para Todos 2024-II}
\fancyhead[R]{\today}

\begin{document}

\maketitle
\begin{flushright}
    Somos polvo de estrellas, y somos una forma\\ de que el universo se conozca a sí mismo.\\ Carl Sagan.
\end{flushright}

\section*{Información del curso}
¡Bienvenidos al curso Astronomía para Todos, esperamos que encuentren útil esta asignatura!. No olviden visitar periódicamente el Google Classroom del curso para acceder al material y estar al tanto de las novedades. Pueden enviarnos sus comentarios mediante el medio de comunicación establecido en el curso (correo institucional, chat classroom).\\
\\
\textbf{Horario:} Martes y jueves 09:00 am - 11:00 am\\
\textbf{Edificio:} Observatorio Astronómico Nacional, salón 116

\section*{Calificaciones}
\begin{itemize}
    \item Talleres y Actividades: 50\%.
    \item Proyecto final: .............25\%.
    \item Asistencia: ...................15\%.
    \item Autocalificación: ..........10\%.
\end{itemize}

Para los talleres, se organizarán grupos de al menos 3 personas (algunos podrían ser de hasta 5 personas, para cubrir la totalidad de estudiantes de forma adecuada). Para las sustentaciones los grupos se seleccionarán de forma aleatoria, y un integrante escogido por el docente hará la sustentación.

\begin{table}[p]
    \centering
    \caption{Contenidos por semana modificación 2025}
    \renewcommand{\arraystretch}{1.2} % Espaciado en las filas
    \begin{tabular}{|c|c|p{20em}|c|}
        \hline
        \textbf{Semana} & \textbf{Tema} & \textbf{Contenido} & \textbf{Fecha} \\
        \hline
        1  & \multirow{4}{*}{Historia} & Presentación del curso & 29/10/2024 \\
           &                           & Nuestro lugar en el universo & 31/10/2024 \\
        \cline{1-1}\cline{3-4}    2  &                           & Astronomía en la historia & 5/11/2024 \\
           &                           & Catálogo de estrellas & 7/11/2024 \\
        \hline
        3  & \multirow{4}{*}{Esfera Celeste} & Movimiento de la Tierra & 12/11/2024 \\
           &                           & Observaciones en el cielo & 14/11/2024 \\
        \cline{1-1}\cline{3-4}    4  &                           & Carta celeste & 19/11/2024 \\
           &                           & Actividad 1: ensayo & 21/11/2024 \\
        \hline
       5  & \multirow{2}{*}{Mecánica Celeste} & Leyes de Kepler - Newton - Cohetes & 26/11/2024 \\
          &                           & \cellcolor[rgb]{ .651,  .788,  .925}Sustentación actividad 1 - máximo 5 minutos por grupo & 28/11/2024 \\
       \hline
       6  & \multirow{2}{*}{OAN} & No Clase:Foro int.Colombia - China & 3/12/2024 \\
          &                           & \cellcolor[rgb]{ .651,  .788,  .925}Visita al OAN sede Campus & 5/12/2024 \\
       \hline
       7  & \multirow{4}{*}{Mecánica Celeste} & Leyes de Kepler - Newton - Cohetes & 10/12/2024 \\
          &                           & Leyes de Kepler - Newton - Cohetes & 12/12/2024 \\
       \cline{1-1}\cline{3-4}    8  &                           & Medio interestelar - Telescopios- Universo en diferentes longitudes de onda & 17/12/2024 \\
          &                           & \cellcolor[rgb]{ .969,  .78,  .675}Fin de año & 19/12/2024 \\
       \hline
       \multicolumn{4}{|c|}{\textbf{Receso Fin de Año}} \\
       \hline
       9  & \multirow{7}{*}{Estrellas} & \cellcolor[rgb]{ .969,  .78,  .675}Inicio de año & 16/01/2025 \\
       \cline{1-1}\cline{3-4}       &                           & \cellcolor[rgb]{ .651,  .788,  .925}Actividad Sistema Solar a escala & 21/01/2025 \\
       10 &                           & Sistema Solar y Fenómenos astronómicos & 23/01/2025 \\
       \cline{1-1}\cline{3-4}       &                           & Sol, otros cuerpos del sistema solar y estrellas & 28/01/2025 \\
       11 &                           & Formación Estelar - Visita Domo AON & 30/01/2025 \\
       \cline{1-1}\cline{3-4}       &                           & \cellcolor[rgb]{ .651,  .788,  .925}Vista OAN sede historica & 4/2/2025 \\
       12 &                           & Astrofotografía - Invitado Andrés Molina & 6/2/2025 \\
       \hline
         & \multirow{2}{*}{Agujeros Negros} & Evolución Estelar & 11/2/2025 \\
       13 &                           & \cellcolor[rgb]{ .651,  .788,  .925}Agujeros Negros- Actividad 2: infografia, fanzin, poster & 13/2/2025 \\
       \hline
          & \multirow{3}{*}{Universo y Estructuras} & Introducción relatividad & 18/2/2025 \\
       14 &                           & \cellcolor[rgb]{ .651,  .788,  .925}Sustentación actividad 2 - máximo 5 minutos por grupo & 20/2/2025 \\
       \cline{1-1}\cline{3-4}       &                           & Formación y expansión del universo - Estructura & 25/02/2025 \\
       \cline{2-4}    15 & \multirow{2}{*}{Cosmología} & Modelo cosmológico & 27/02/2025 \\
       \cline{1-1}       &                           & Futuro del universo & 4/3/2025 \\
       \cline{2-4}    16 & \multirow{2}{*}{Finalización del Curso} & \cellcolor[rgb]{ 1,  1,  0}Proyecto final & 6/3/2025 \\
       \cline{1-1}\cline{3-4}       &                           & Reporte de calificaciones SIA & 10/3/2025 \\
       \hline
   \end{tabular}
   \label{tab:contenido_semanal}
\end{table}


\section*{Referencias}

\begin{itemize}
    \item H. Karttunen, P. Kröger, H. Oja, M. Poutanen, y K. J. Donner, \textit{Fundamental Astronomy}. Springer, 6a ed., 2017.
    \item B. W. Carroll y D. A. Ostlie, \textit{An Introduction to Modern Astrophysics}. Cambridge, 2a ed., 2017.
    \item P. Jain, \textit{An Introduction to Astronomy and Astrophysics}. CRC, 1a ed., 2015.
    \item J. G. Portilla, \textit{Astronomía para todos}. Universidad Nacional de Colombia, 2a ed., 2001.
    \item J. G. Portilla, \textit{Elementos de astronomía de posición}. Universidad Nacional de Colombia, 1a ed., 2012.
    \item M. Begelman y M. Rees, \textit{Gravity’s Fatal Attraction: Black Holes in the Universe}. Cambridge, 2a ed., 2010.
    \item J. Wikilson, \textit{The Solar System in Close-Up}. Springer, 1a ed., 2016.
\end{itemize}

\section*{Aplicaciones recomendadas}

Es importante conocer el uso de los recursos electrónicos y repositorios que tiene la universidad para acceder a información científica.
\begin{itemize}
    \item Sistema Nacional de Bibliotecas (\url{https://bibliotecas.unal.edu.co/})
    \item Recursos Electrónicos (\url{https://login.ezproxy.unal.edu.co/menu})
\end{itemize}
\begin{itemize}
    \item Stellarium (\url{https://stellarium.org/es/})
    \item Aladin Sky Atlas (\url{http://aladin.u-strasbg.fr/})
    \item SAOImageDS9 (\url{https://sites.google.com/cfa.harvard.edu/saoimageds9?pli=1})
    \item TOPCAT (\url{http://www.star.bris.ac.uk/~mbt/topcat/})
    \item Cartas celestes (\url{https://astroaficion.com/2013/03/25/cartas-celestes/})
\end{itemize}

\end{document}
