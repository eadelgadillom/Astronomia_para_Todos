\documentclass[12pt]{article}
\usepackage[utf8]{inputenc}
\usepackage[spanish]{babel}
\usepackage{geometry}
\geometry{a4paper, margin=1in}
\usepackage{enumitem}
\setlist{nosep}

\title{Actividad 1: Análisis de la película "Contacto"}
\author{Astronomía para Todos \\ Grupo 2 - 2024-2}
\date{}

\begin{document}

\maketitle

\section*{Objetivo de la Actividad}

Observar la película \textit{Contacto}, la cual se encuentra en el classroom del curso, basada en el libro homónimo de Carl Sagan, con el propósito de comprender la aplicación y uso de los sistemas de coordenadas, constelaciones, magnitudes, estrellas, distancias e introducir conceptos como técnicas observacionales y longitud de onda.

\section*{Instrucciones}

\begin{enumerate}
    \item Realizar una caracterización de la o las estrellas u objetos mencionados en la película. Esta caracterización debe incluir:
    \begin{itemize}
        \item Nombre
        \item Historia
        \item Región
        \item Magnitud
        \item Visibilidad desde la tierra (tiempo visible en el año)
        \item Distancia desde la Tierra
        \item Coordenadas ecuatoriales geocéntricas
        \item Coordenadas horizontales para el dia 21 de noviembre de 2024 a la hora 12:00 + (numero de grupo)
        \item Paralaje
    \end{itemize}
    \item Elaborar un análisis y discusión en formato ensayo, con una extensión mínima de 2 páginas, abordando algunos de los siguientes enfoques:
    \begin{itemize}
        \item El papel de la ciencia y la investigación en la sociedad actual.
        \item Vida en el universo.
        \item Programa SETI.
        \item Distancias en el universo.
        \item Telescopios.
        \item Conflictos religiosos, politicos, éticos, económicos y sociales.
        \item Otros aspecto relevante que considere el grupo.
    \end{itemize}
\end{enumerate}

\section*{Formato de Entrega}

\begin{itemize}
    \item El trabajo debe ser entregado en formato PDF.
    \item La caracterización de las estrellas debe incluir tablas, imágenes y referencias.
    \item El ensayo debe estar redactado con claridad y en formato de ensayo.
    \item Sustentación del trabajo: 28 de noviembre de 2024 (5 - 10 minutos por grupo).
\end{itemize}


\end{document}

