\documentclass[12pt]{article}
\usepackage[utf8]{inputenc}
\usepackage[spanish]{babel}
\usepackage{geometry}
\usepackage{hyperref}
\geometry{a4paper, margin=1in}
\usepackage{enumitem}
\setlist{nosep}

\title{Proyecto Final}
\author{Astronomía para Todos \\ Grupo 2 - 2024-2}
\date{}

\begin{document}

\maketitle

\section*{Objetivo de la Actividad}

Reconocer de manera introductoria, en qué consiste el ejercicio practico de un astrónomo. Para esto se recomienda escoger un tema y desarrollar un pequeño proyecto, el tema del proyecto es libre, sin embargo, a continuación se adjunta una lista de proyectos sugeridos como actividad final del curso de Astronomía para Todos.

\section*{Proyectos}

\begin{enumerate}
    \item \textbf{Estimación de elementos orbitales de cuerpos menores empleando la librería Rebound}:\\Estudio de órbita real de un cuerpo en el sistema Solar. Este proyecto es ideal para estudiantes que estén familiarizados con la programación. Se recomienda revisar las rutinas mas generales de la librería Rebound \url{https://rebound.readthedocs.io/en/latest/}. En el Classroom del curso, en el apartado material de apoyo se adjunta un ejemplo y material de apoyo respecto a esta actividad.
    \item \textbf{Observación de disco solar}:\\ Explicacion y estudio de la actividad solar durante al menos 3 dias de observación. Las observaciones en general pueden tomarse de la fotografía actualizadas del Solar Dinamic Observatory (SDO) en \url{https://sdo.gsfc.nasa.gov/data/}, por ejemplo el estudio de las manchas solares.
    \item \textbf{Estudios de espectros de estrellas}:\\De las bases de datos de espectros de estrellas \url{https://www.eso.org/sci/observing/tools/uvespop/interface.html}, escoger al menos 5 que tengan rasgos diferentes, pueden ser distintas clases espectrales y determinar a partir del espectro, algunas de sus características más generales, temperatura, composición, emtre otros.
    \item \textbf{Caracterizar una region  en diferentes longitudes de onda}:\\Observar una región de interés en los diferentes rangos posibles, de acuerdo con las bases de datos disponibles en el atlas astronómico Aladin \url{https://aladin.cds.unistra.fr/AladinLite/} o los distintos telescopios  disponibles en \url{https://skyview.gsfc.nasa.gov/current/cgi/query.pl}. Hacer una caracterización de la región de acuerdo con la información obtenida de las estrellas u otros objetos en la región escogida.
    \item \textbf{Alimentación de astronautas en el espacio}:\\Utilizar documentos y estudios de referencia en revistas científicas sobre cómo es el proceso de alimentación en el espacio y que implicaciones tiene en el cuerpo humano. Para esto se puede hacer uso de el acceso a las revistas científicas que se pueden ingresar por el repositorio de la universidad \url{https://bases.unal.edu.co/}.
    \item \textbf{Efecto de la radiación espacial en materiales}:\\Investigar cómo la exposición prolongada al ambiente espacial, la radiación y a las condiciones del medio interestelar afecta materiales en satélites y las distintas misiones.
    \item \textbf{Explicación articulo científico}:\\Seleccionar un articulo científico de interés, replicarlo y explicarlo en clase. Se recomienda consultar en \url{https://ui.adsabs.harvard.edu/} artículos de algún tema astronómico o astrofísico.
\end{enumerate}

\section*{Formato de Entrega}

\begin{itemize}
    \item Una presentación organizada para los estudiantes del curso, en donde se expongan todos los detalles relacionados con el proyecto escogido. La entrega de un documento escrito es opcional, pero se sugiere hacerlo para organizar mejor las ideas antes de la presentación del proyecto.
    \item Sustentación del trabajo: 6 de Marzo de 2024 (5 - 10 minutos por grupo).
\end{itemize}


\end{document}

