\documentclass[12pt]{article}
\usepackage[utf8]{inputenc}
\usepackage[spanish]{babel}
\usepackage{geometry}
\geometry{a4paper, margin=1in}
\usepackage{enumitem}
\setlist{nosep}

\title{Actividad 2: Apreciación de la película "Interestelar"}
\author{Astronomía para Todos \\ Grupo 2 - 2024-2}
\date{}

\begin{document}

\maketitle

\section*{Objetivo de la Actividad}

Observar la película \textit{Interestelar}, la cual se encuentra en el classroom del curso, con el propósito de comprender los fenómenos asociados a la relatividad general  y los objetos compactos como los agujeros negros.
La presentación de los poster se hará tipo congreso cientifico.

\section*{Instrucciones}

\begin{enumerate}
    \item Realizar un elemento visual gráfico tipo fanzine, poster, infografia, flyer en formato vertical. Este elemento debe incluir:
    \begin{itemize}
        \item Titulo llamativo.
        \item Tema novedoso e interesante.
        \item Contexto general.
        \item Representación en la película.
        \item El poster debería poder explicarse por si solo, debe ser claro.
        \item Las imágenes, o textos deben incluir referencias.
        \item Otros aspectos relevantes que considere el grupo.
    \end{itemize}
\end{enumerate}

\section*{Formato de Entrega}

\begin{itemize}
    \item El trabajo debe ser entregado en formato PDF en el classroom del curso.
    \item Sustentación del trabajo: 20 de noviembre de 2024 (5 - 10 minutos por grupo).
\end{itemize}


\end{document}

